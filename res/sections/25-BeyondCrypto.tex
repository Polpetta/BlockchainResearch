\section{Beyond Cryptocurrencies}
\label{sec:beyond_crypto}

Thanks to its design, Blockchain is finding different applications. Here
we will briefly show and discuss about the other places where Blockchain
is being deployed.

\paragraph*{Internet of Things}

Recently, Blockchain has found an application in the IoT\footnote{IoT stands
for \textit{Internet of Things}} world\cite{politecnico16}. Its characteristic
allows to build a decentralized system where IoT devices can store information,
without the possibility from an attacker to change the data already saved in
the database\cite{politecnico16}.

The main problem of using Blockchain as a P2P data storage in a IoT environment
is that if the network is too small the possibility for an attacker of taking
control of the main chain are high using a Bitcoin-similar proof-of-work. In
fact, the Blockchain is more vulnerable when the number of miners is small and
when the blockchain is new. A way to mitigate this issue could be allowing only
a restricted or trusted nodes to mine transactions. This seems to lead to
centralization, but it makes sense if the blockchain is private (the access is
restricted to a close number of people) and the IoT devices can not be trusted.
It is important to take in consideration that centralizing the mining nodes
attackers will focus on those, trying to inject malware or viruses to be able to
take control of the chain. Restricting the mining nodes means that is important
to pay particular attention to these nodes.
Moreover, the mining process could be public if the Blockchain does not contain
confidential information as explained in\cite{tian17}, where a combination of
RFID, IoT and Blockchain technology can lead to the monitoring of a food supply
chain.

Another solution to this problem could be using an already existing and mature
blockchain, as the Bitcoin one, with the disadvantage of having to parse only
the data sent by other IoT devices. An attacker with this solution instead of
trying to take over the whole blockchain it could try to forge false data,
pushing the transaction in the same chain (since the Bitcoin blockchain is
public and anyone can make new transactions) and make the other nodes believe
that the data is authentic.

\paragraph*{Access Control Manager}
\label{bc:acm}

As proposed in\cite{dp15}, Blockchain can act as an access manager to enhance
users privacy in modern services. This could be very useful especially in the
mobile area, where applications can have easy access to personal data. In the
system design, the authors take in consideration to use a local database
(LevelDB) to be able to save data in a key-store way. This is a demostrantion
that even though Blockchain can be used as access control manager there is the
need to save some data in a centralized way, thus allowing attackers to use
well known attacks/vulnerabilities (e.g. DDoS, Sniffing).

\paragraph*{Storing records}

Blockchain is also able to attach data up to 1Mb \cite{ectel16} inside a
transaction, allowing public administrations to use Blockchain to store
records, such as public reports. For instance in\cite{ectel16} the authors
proposed to save papers inside a Blockchain database, in order to keep it
ordered by timestamps and accessible to everyone, thanks to the P2P networking.
Allowing everyone to upload data in transactions could be a threat, because this
allow attackers to send malware or auto-runnable scripts to infects the nodes,
and proper file checking is indeed necessary before saving them in the chain.


Recently different projects have tried to create a NoSQL-like database with
Blockchain, but these projects (we cite for example BigchainDB\footnote{The
project webpage is \url{https://www.bigchaindb.com/}}) have to rely on a
centralized database, suffering the same problems described in~\ref{bc:acm}.
